\documentclass[english]{beamer} %,handout
\usepackage{amsmath}
\usepackage{graphicx}

\makeatletter

\usepackage{listings}
\usetheme{Boadilla}

\setbeamercovered{transparent}

\usecolortheme{purdue}

\usepackage{babel}

\begin{document}

\title[HMMs \& CRFs]{Hidden Markov Models \&\\
Conditional Random Fields}

\author{Andrei Barbu}
\institute[Purdue University]
{
  School of Electrical and Computer Engineering\\
  Purdue University
}

\begin{frame}
  \titlepage
  \begin{center}
    \includegraphics[scale=0.6]{PU_signature_eps}
  \end{center}
\end{frame}

\newcommand{\Gaussian}{\rput(0,-0.35){\psset{yunit=0.8cm,xunit=0.3}
     \psGauss[linecolor=red, linewidth=0.8pt, sigma=0.5]{-1.5}{1.5}}}
\def\dedge{\ncline[linestyle=dashed]}
\def\omitnode{\Tr*[edge=\dedge]{}}

\begin{frame}[<+->]{History}
\begin{itemize}
\item Developed by Markov in 1906
\item Markov was a disciple of Chebyshev  along with Lyapunov
\item Introduced for no practical reason, except maybe to spite Nekrasov\\
due to the dispute over the Weak Law of Large Numbers
\item Within 1 year it was being used to clear up issues in thermodynamics
\end{itemize}
\end{frame}

\begin{frame}[<+->]{Markov Models}
\begin{itemize}
\item Process with the Markov property: $P(s_{t+1}|s_{t},...,s_{0})=P(s_{t+1}|s_{t})$
\item Process with discrete time\end{itemize}
\begin{uncoverenv}%{}
<3->
\begin{exampleblock}
{Two processes}
\begin{itemize}
\item $\left[\begin{array}{cc}
p_{00} & p_{01}\\
p_{10} & p_{11}\end{array}\right]$
\item $\sum_{i}p_{ji}=1$
\item  $\left[\begin{array}{cc}
p_{00} & p_{01}\\
p_{10} & p_{11}\end{array}\right]\left[\begin{array}{c}
N_{0}(t)\\
N_{1}(t)\end{array}\right]=\left[\begin{array}{c}
N_{0}(t+1)\\
N_{1}(t+1)\end{array}\right]$
\end{itemize}
\end{exampleblock}
\end{uncoverenv}%{}
\end{frame}

\end{document}
